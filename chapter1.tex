% !TeX root = document.tex
% !TeX encoding = UTF-8 Unicode

\chapter{Introduction}%
\label{chp:introduction}

The fundamental frameworks of classical and modern control theory are conceived
for unconstrained systems. Therefore, if constraints are of interest, ad-hoc
solutions must used. Some approaches handle saturating actuators by using a
polytopic representation or generalized sector
conditions~\parencite{tarbouriech.garcia.ea:stability}. However, these
techniques cannot handle the state's constraints directly, requiring the
inclusion of state-space constraints~\parencite{klug.castelan.ea:fuzzy}. Some
techniques started to be developed that enforces constraints, like the model
predictive control (MPC)~\parencite{wang:model,zhang:fast}. However, they
require the design of a new controller following the framework. Another line of
research tried to solve the same problem by using existing controllers. Instead
of trying to compute the optimal control path that keeps the system constrained,
it uses model prediction to change the reference given to existing controllers
to keep the system constrained. The first of such techniques were reference
filters that also imposed only
soft-constraints~\parencite{vahidi.kolmanovsky.ea:constraint}.

The idea then evolved into Reference Governors (RG), which makes use of
optimization to find the best reference to
follow~\parencite{gilbert.kolmanovsky.ea:discrete‐time}. The RG gives a virtual
reference \(g(k)\) to the controller, which is a scaled version of the real
reference \(r(k)\), read \(g(k)=\delta r(k)\). The optimization problem finds the
best \(\delta r(k)\) that minimizes the distance between \(g(k)\) and \(y(k)\)
without violating the constraints. Because of the simplicity of the optimization
problem, this approach has an easy implementation but suffers from loss of
dimensions. Such a loss comes from the fact that variation is one-dimensional
while \(r(k)\) may be multidimensional~\parencite{gilbert.kolmanovsky:fast}.

Building on this idea,~\textcite{bemporad.casavola.ea:nonlinear}
and~\textcite{bemporad.casavola.ea:nonlinear} developed what is known today as
the \ac{CG} approach. The difference to RG is that the CG technique optimizes
\(g(k)\) directly, requiring more computational processing power but yielding
better system performance, especially when the reference has dimension greater
than 1. They also built on the work of~\textcite{kapasouris.athans.ea:design},
which explores the ideas of the Lyapunov Theorem and Invariant Sets Theorem.

Switching systems are composed of many subsystems, called modes, which switch
according to a switching
rule~\parencite{liberzon:switching,liberzon.morse:basic}. Only one subsystem can
be active at a given time. The switching can cause instability even when all
subsystems are stable, which leads to the notion of a dwell-time: how long a
subsystem must remain active after switching to avoid
instability~\parencite{liberzon.morse:basic}. Different approaches have been
proposed to compute the minimum dwell time
(see~\parencite{chesi.colaneri.ea:computing} and reference therein) and
stabilizing controller (see~\parencite{lin.antsaklis:stability} for switched
linear systems). Fewer solutions exist to deal with constrained switching
systems, see e.g.~\parencite{franzè.lucia.ea:command,lucia.franzè:stabilization}
and references therein.

In \parencite{franzè.lucia.ea:command,lucia.franzè:stabilization}, the CG
framework is used to supervise the system mode switches and assure both
stability and constraint satisfaction.

\textbf{Notations:} The set of real numbers is denoted by \(\mathbb{R}\).
\(\mathbb{R}^n\) denotes a vector of \(n\) real elements. \(\mathcal{V}\),
\(\mathcal{W}\) and \(\mathcal{C}\) represent sets. \ac{AT} represents the
conjugate transpose of \(A\). \(g(k)\) represents a discrete-time signal. We
define \(x^\top\Psi x\) with \(x\in\mathbb{R}^n\) and semidefinite positive matrix
\(\Psi \in \mathbb{R}^{n \times n}\) as the norm \(\norm{x}^2_\Psi{}\).

\section{Objectives}%
\label{sec:objectives}

In this work, considering the class of switched systems (switching systems with
controlled switching signals), we propose a novel switching rule based on the
switching CG's region of attraction. It aims at improving the control
performance obtained in the previous work of~\textcite{franzè.lucia.ea:command}
where mode-switches occur after a worst-case dwell-time elapses.

The presented approach consists of a two-step switch, which uses an intermediate
CG.\ This intermediate CG has the constraints of the source CG and the
controller of the target CG.\ To change from \(CG_i\) to \(CG_j\), you first
switch from \(CG_i\) to a \(CG_{ij}\) when the system enters the region of
attraction of \(CG_{ij}\)'s controller, and then changes to \(CG_j\) when the
system enters the intersection of the constraint regions between \(CG_{ij}\) and
\(CG_j\). Thus, if it is verified that the controller of \(CG_j\) has better
performance than that of \(CG_1\), it can be used from an earlier stage.

\section{Document organization}%
\label{sec:organization}

The main concepts involved in this work are explained in
Chapter~\ref{chp:theoretical-foundations} -
\nameref{chp:theoretical-foundations}.
Chapter~\ref{chp:practival-implementation-aspects} -
\nameref{chp:practival-implementation-aspects} discusses some practical aspects
of implementing the command governor. In Chapter~\ref{chp:results} -
\nameref{chp:results} we show two experiments that illustrate the advantages of
the proposed technique.
