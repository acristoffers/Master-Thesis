% !TeX root = document.tex
% !TeX encoding = UTF-8 Unicode

\chapter{Introduction}%
\label{chp:introduction}

Physical systems have constraints on their inputs, outputs, and states, which
are usually ignored to facilitate the controller's design. Such constraints can
result from physical limitations, such as a tank's capacity or a turbine's
maximum power output, or can be mathematically imposed to achieve some goal,
like keeping a process' pH bounded. In the literature, different solutions
exists to address actuator saturation constraints, see
e.g.~\parencite{klug.castelan.ea:fuzzy,tarbouriech.garcia.ea:stability}.
However, those techniques do not enforce constraints but design controllers that
avoid the borders of constraints or even allow the system to violate them for a
short period.

By taking advantage of optimization techniques, Model Predictive Control uses an
online optimization procedure to predict a system's state evolution on a
prediction horizon and compute a system's optimal control trajectory and allows
constraints to be imposed to states, inputs, outputs, and their
variations~\parencite{wang:model,zhang:fast}. As the optimization procedure is
online, constraints are enforced, and the controller will actively avoid
violations while allowing for less conservative control paths.

Besides MPC, other techniques were developed to keep systems constrained, but
instead of computing the optimal control path that keeps the system constrained,
it uses model prediction to change the reference given to existing controllers
to keep the system constrained. The first of such techniques were reference
filters, which imposed only
soft-constraints~\parencite{vahidi.kolmanovsky.ea:constraint}, i.e. constraints
which penalize the objective function when constraints are violated instead of
rendering the problem infeasible.

This idea then evolved into Reference Governors (RG), which divides the problem
into two parts: tracking and constraint enforcement. The controller solves the
tracking problem in the inner loop without taking constraints into account, and
the Reference Governor solves the constraint enforcement problem in the outer
loop by using the reference and system's output to change the inner loop's
reference. Figure~\ref{fig:rg-block-diagram} shows a block diagram for this
technique. The optimization problem finds the best \(\delta\) that minimizes the
distance between \(g(k)\) and \(r(k)\) without violating constraints. Because of
the numerical simplicity of the optimization problem, this approach has an easy
implementation but suffers from loss of dimensions. Such a loss comes from the
fact that \(\delta\in\mathcal{R}\) while
\(r(k)\in\mathcal{R}^n\)~\parencite{gilbert.kolmanovsky:fast}.

\begin{figure}
  \centering
  \tikzstyle{block}    = [draw, rectangle, minimum height=3em, minimum width=6em]
  \tikzstyle{sum}      = [draw, circle, node distance=1cm]
  \tikzstyle{point}    = [coordinate]
  \tikzstyle{pinstyle} = [pin edge={to-,thin,black}]
  \begin{tikzpicture}[auto, node distance=2cm,>=latex']
    % We start by placing the blocks
    \node [point, name=input] {};
    \node [sum] (prod) [right=4cm of input] {\Large \(\times\)};
    \node [block] (controller) [right=1cm of prod] {Controller};
    \node [block, above of=prod] (rg) {Reference Governor};
    \node [block] (system) [right=1cm of controller] {System};
    \node [point, name=output, node distance=3cm, right of=system] {};
    \node [point, name=state, above of=system] {};
    \node [point, name=ref, left of=rg] {};

    \draw [draw,->] (input) -- node [name=r] {\(r(k)\)} (prod);
    \draw [] (r) |- (ref);
    \draw [->] (ref) -- (rg);
    \draw [->] (rg) -- node {\(\delta\)} (prod);
    \draw [->] (prod) -- node {\(g(k)\)} (controller);
    \draw [->] (controller) -- node {\(u(k)\)} (system);
    \draw [->] (system) -- node {\(y(k)\)} (output);
    \draw [] (system) -- node {\(x(k)\)} (state);
    \draw [->] (state) -- node {} (rg);
    \draw [->] (state) -| node {} (controller);
  \end{tikzpicture}
  \caption[Reference Governor's block diagram for a linear system.]{Reference
    Governor's block diagram for a linear system. The inner loop's controller
    guarantees reference tracking and the outer loop guarantees constraints
    satisfaction.}%
  \label{fig:rg-block-diagram}
\end{figure}

Building on this idea and on the work of~\textcite{kapasouris.athans.ea:design},
which explores the ideas of the Lyapunov Theorem and Invariant Sets
Theorem~\parencite{blanchini.miani:set-theoretic},
\textcite{bemporad.casavola.ea:nonlinear}
and~\textcite{casavola.mosca.ea:robust} developed what is known today as the
\acf{CG} approach. The difference is that Reference Governors optimize a number
\(\delta\), which multiplies the reference \(r(k)\) to create the virtual reference
\(g(k)\), whereas Command Governors optimize the the vector \(g(k)\) directly,
requiring more computational processing power but yielding better closed-loop
performance.

Reference and Command Governors are still subjects of studies, and used in
conjunction with other techniques. It has been of particular interest on the
adaptive control field
\parencite{arabi.yucelen.ea:command,ristevski.dogan.ea:transient,wilcher.jaramillo.ea:on,dogan.yucelen.ea:improving,gruenwald.yucelen.ea:expanded,makavita.jayasinghe.ea:experimental}
as a mean to add constraints to the system. It has also been used to constrain
switching networks \parencite{ong.djamari.ea:governor}, networks with delays
\parencite{shen.song.ea:constrained} and interconnected systems
\parencite{tedesco.casavola:turn-based}. \textcite{peng.wang.ea:constrained}
also used a Command Governor to develop an anti-disturbance controller for an
uncertain, constrained system
and~\textcite{schwerdtner.bortoff.ea:projection-based} developed an anti-windup
controller for systems with saturated inputs.
\textcite{seeber.golles.ea:reference} provide a real-time implementation of
reference shaping for a biomass grate boiler, based on Command Governor, to
avoid actuator saturation and to constrain mass-flow.

The Command Governor can be used in so many different scenarios because it is a
add-on technique, not requiring adaptation from the controller or system. This
is one main advantage of the Command Governor, as well as the fact that it
enforces hard-constraints. The main drawback is that it requires an online
optimization procedure, which might not be doable on systems with fast dynamics.
Another concern is that the optimization problem might not be feasible in the
presence of model uncertainties, as, for example, an observer might put the
model's state on an invalid state for the optimization problem. A common
technique is to re-apply the last calculated reference value when the
optimization problem is unfeasible, but it can lead to instability if the
situation persists for extended periods of
time~\parencite{garone.di-cairano.ea:reference}.

This work applies the \CG{} technique to switched systems. Switched systems are
composed of many subsystems, called modes, which switch according to a switching
rule~\parencite{liberzon:switching,liberzon.morse:basic}. Only one subsystem can
be active at a given time. The switching can cause instability even when all
subsystems are stable. Techniques exist to guarantee stability under abirtrary
switching, such as using polytopic linear parameter varying
representations~\parencite{deaecto.geromel.ea:robust}, but they are extremely
conservative. Development leads to the notion of a dwell-time: how long a
subsystem must remain active before switching to avoid
instability~\parencite{liberzon.morse:basic}. Different approaches have been
proposed to compute the minimum dwell time
(see~\parencite{chesi.colaneri.ea:computing} and reference therein) and
stabilizing controller (see~\parencite{lin.antsaklis:stability} for switched
linear systems). Fewer solutions exist to deal with constrained switching
systems, see e.g.~\parencite{franzè.lucia.ea:command,lucia.franzè:stabilization}
and references therein.

In \parencite{franzè.lucia.ea:command,lucia.franzè:stabilization}, the \CG{}
framework is used to supervise the system mode switches and to assure both
stability and constraint satisfaction.

\section{Objectives}%
\label{sec:objectives}

In this work, considering a class of constrained switched systems, we propose a
novel switching rule based on the controllers's region of attraction. There are
other techniques for stabilizing switched systems, but for constrained switched
systems, specially when using the Command Governor structure, the dwell-time
seems to be the only one applied. Without employing the Command Governor
structure, however, the Model Predictive Control technique accomplishes the same
goal. For the proposed technique two rules can be stated:

\begin{itemize}
  \item the system's state is inside the next controller's region of attraction.
  \item the system's state is inside the command governors' constraint's
        intersection.
\end{itemize}

This allows two switching rules. In the first scenario, the first rule triggers
a partial switch, in which only the controller is changed. The second rule is
used to complete the switch, swaping the active command governor unit. We call
this a hybrid switch. The second scenario makes a complete switch using only the
second rule. Both scenarios need the same stability condition, the only
difference is that the second avoids the first set membership check, at the cost
of possibly missing on convergence's speed gains opportunities.

\section{Thesis organization}%
\label{sec:organization}

The main concepts involved in this work are explained in
Chapter~\ref{chp:theoretical-foundations} -
\nameref{chp:theoretical-foundations}. The proposed technique is explained in
Chapter~\ref{chp:switching-rules} - \nameref{chp:switching-rules}. In
Chapter~\ref{chp:results} - \nameref{chp:results} we show two experiments that
illustrate the advantages of the proposed technique.
