% !TeX root = document.tex
% !TeX encoding = UTF-8 Unicode

\begin{abstract}
  Physical systems are subject to constraints, like input saturation or physical
  limits to the states, which also appears on each switched system mode. One way
  of dealing with them is to apply the Command Governor scheme, which changes
  the controller's reference to enforce constraints. When the system is a
  switched one, the problem of stability arises. Switched systems are a common
  kind of system used to describe non-linear systems by dividing them into
  linear sections or different modes of operations of the same system, like the
  different phases an airplane goes through during take-off or landing. However,
  arbitrarily switching the modes of a switched system can cause instability,
  requiring a switching rule design. The most commonly used rule is the
  dwell-time switch, in which the system waits for a dwell-time to elapse after
  the references changes to switch modes. Seeing the possibility of speeding up
  such systems' convergence, we propose a new rule based on the controller's
  Region of Attraction, which requires the system's state to be inside the
  mode's controller's Region of Attraction to switch, guaranteeing stability
  after a mode switch. With this technique, we also propose a hybrid switch
  technique, which can further speed up convergence and generate lower actuator
  effort in some cases. We present some simulations to illustrate the proposal's
  potential and compare it with a scheme exploiting a dwell-time approach. The
  results suggest that our approach adds new CG and supervisor design
  possibilities, reducing the transition time between system modes and improving
  the closed-loop performance indexes.
\end{abstract}

\textbf{Keywords}: Command Governor, Discrete-time systems, Switching systems,
Lyapunov stability, Region of attraction
