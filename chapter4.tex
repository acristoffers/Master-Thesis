% !TeX root = document.tex
% !TeX encoding = UTF-8 Unicode

\chapter{Results}%
\label{chp:results}

This chapter presents both simulations and experimental results that show the
technique working and compares it with dwell-time implementations.

\section{Simulations}%
\label{sec:simulation}

This section presents three MIMO-systems simulations using both region of
attraction techniques described in Algorithm~\ref{alg:roa-rule} (with and
without early controller switching). The level-system presented models a
physical system present in our laboratory, the unstable system is purely
theoretical, and the Cessna 182 was taken from an article for comparison.

\subsection{Level Control System}%
\label{subsec:tanks-system}

Consider an interactive tank system as indicated in Figure~\ref{fig:tanks-sim}.
It consists of two coupled tanks, namely \(T1\) and \(T2\), that are feed by two
with controlled outflows \(u_1\) and \(u_2\), measured in
\si{\cubic\centi\metre\per\second}. The levels of each tank, \(h_1\) and \(h_2\)
(\si{\centi\metre}), are the control objective variables.

\input{imgs/tanks-sim}

The output flow of the \(T1\) and \(T2\) are denoted by \(q_1\) and \(q_2\)
(\si{\cubic\centi\metre\per\second}), respectively, and the flow between them is
noted by \(q_{12}\) (\si{\cubic\centi\metre\per\second}).

Both tanks have the same cross-section area, denoted as \(A\)
(\si{\square\centi\metre}), as well as the cross-section areas of the restrictions in
the outputs of the tanks, \(a\) (\si{\square\centi\metre}); \(g\)
(\si{\centi\metre\per\square\second}) is the gravity acceleration. By using
Bernoulli's equations, we have:
%
\begin{equation}
  \label{eq:formula-height-variation-lin}
  \begin{aligned}
    \dot{h}_1(t) & = \frac{u_1(t)-q_1(t)\pm{}q_{12}}{A}, \\
    \dot{h}_2(t) & = \frac{u_2(t)-q_2(t)\mp{}q_{12}}{A},
  \end{aligned}
\end{equation}
%
where the flows are given by \(q_1(t) = a\sqrt{2gh_1(t)}\),
\(q_2(t) = a\sqrt{2gh_2(t)}\), and
\(q_{12}(t) = a\sqrt{2g\left|h_2(t)-h_1(t)\right|}\).

This is a nonlinear and switching system, as the model changes depending on the
height of the tanks' water column. At each mode, \(h_1>h_2\) or \(h_1\leq h_2\),
equation~\eqref{eq:formula-height-variation-lin} can be linearized around an
equilibrium operational point \((x_{eq},u_{eq})\) by using Jacobian matrices. In
what follows, we use \(a=\SI{5.9}{\centi\metre\squared}\),
\(A=\SI{961\pi{}}{\centi\metre\squared}\), and
\(g=\SI{980.665}{\centi\metre\per\square\second}\).

Two operational conditions with
\(x(k) = \begin{bmatrix}h_1(k) & h_2(k)\end{bmatrix}^\top{}\), chosen so one tank
is nearly full, are such that:
%
\begin{equation*}
  x_{eq}^1 =\! \begin{bmatrix}
    57.5 \\ 43.61
  \end{bmatrix}\!,
  u_{eq}^1 =\! \begin{bmatrix}
    744 \\ 2960
  \end{bmatrix}\!,
  x_{eq}^2 = \begin{bmatrix}
    43.61 \\ 57.5
  \end{bmatrix}\!,
  u_{eq}^2 =\! \begin{bmatrix}
    2960 \\ 744
  \end{bmatrix}\!,
\end{equation*}
%
yielding two operational modes, each of them corresponding to a \CG{}.\@
%
After the linearization, we discretized the continuous-time model using a sample
time of \SI{5}{\second} and Euler equations to get discrete-time model given
by~\eqref{eq:state-space} with matrices:
%
\begin{align*}
  A_1 & =
  \begin{bmatrix}
    0.92  & 0.053 \\
    0.053 & 0.91
  \end{bmatrix}, ~~~ A_2 = \begin{bmatrix}
    0.91  & 0.053 \\
    0.053 & 0.92
  \end{bmatrix} \\
  B_1 & =
  \begin{bmatrix}
    0.0016           & 4.5\times10^{-5} \\
    4.5\times10^{-5} & 0.0016
  \end{bmatrix},
  B_2 = \begin{bmatrix}
    0.0016           & 4.5\times10^{-5} \\
    4.5\times10^{-5} & 0.0016
  \end{bmatrix},                              \\
  %
  C_1 & = C_2 =
  \begin{bmatrix}
    1 & 0 \\
    0 & 1
  \end{bmatrix},
\end{align*}
%
and number of modes \(s=2\). Using Lyapunov's
approach~\parencite{chen:linear,hespanha:linear} we got the following
controller:
%
\begin{align*}
  K_1 & = \begin{bmatrix}
    -875.384 & -9.217   & -297.447 & 7.982    \\
    -8.505   & -849.853 & 8.514    & -279.434
  \end{bmatrix}, \\
  %
  K_2 & = \begin{bmatrix}
    -849.853 & -8.505   & -279.434 & 8.514    \\
    -9.217   & -875.384 & 7.982    & -297.447
  \end{bmatrix},
\end{align*}
%
for the operational modes \(1\) and \(2\).

We simulated our approaches described by Algorithm~\ref{alg:roa-rule} to switch
between the \CG{}s 1 and 2. In both cases the system starts in \(x(0) =
\begin{bsmallmatrix}43.61 & 57.5\end{bsmallmatrix}^{\top}\). The references are
\(r = \begin{bsmallmatrix}57.5 & 43.61\end{bsmallmatrix}^{\top}\), for
\(0\leq k\leq 10 \cup 50 \leq k \leq\infty{}\), and
\(r = \begin{bsmallmatrix}43.61& 57.5\end{bsmallmatrix}^{\top}\), for
\(11 \leq k \leq 49\). Therefore, the system must perform a closed path. Note that the
references are the system's two equilibrium points, so the systems starts in
mode 1, moves to mode 2 and goes back to mode 1.

The regions of attraction were estimated using the optimization procedure
presented in Section~\ref{sec:region-of-attraction}. The following inequalities
define each mode's regions of attraction:
%
\begin{align}
  \mathcal{L}_{V_{1}}(x) & = x^{\top} \begin{bmatrix}
    4.766\times{}10^{-3} & 1.431\times{}10^{-8} \\
    1.431\times{}10^{-8} & 4.766\times{}10^{-3}
  \end{bmatrix}x \leq 1 \\
  \mathcal{L}_{V_{2}}(x) & = x^{\top} \begin{bmatrix}
    4.766\times{}10^{-3} & 1.185\times{}10^{-8} \\
    1.185\times{}10^{-8} & 4.766\times{}10^{-3}
  \end{bmatrix}x \leq 1
\end{align}

The constraint regions are, in this case, also defined as ellipses, given by the
following inequalities:
%
\begin{align}
  \mathcal{C}_{1}(x) & = \frac{(x_{1}-50.55)^{2}}{8.681^{2}} + \frac{(x_{2}-43.61)^{2}}{1.085^{2}} \leq 1 \\
  \mathcal{C}_{2}(x) & = \frac{(x_{1}-43.61)^{2}}{1.085^{2}} + \frac{(x_{2}-50.55)^{2}}{8.681^{2}} \leq 1
\end{align}
%
where \(x=\begin{bsmallmatrix} x_{1} & x_{2} \end{bsmallmatrix}^{\top}\).

Since the system is linearized around two different operation points, there are
three bases, and the states need to be converted between them. The first is the
basis of the non-linear system, called \enquote{global}. The second and third
are those of the linearized systems. The constraints are written on the global
basis and the regions of attraction in the respective mode's basis. Care must be
taken to convert between the basis for set membership tests.

Fig.~\ref{fig:states} shows the path taken by the closed-loop system over the
space of the system's state. The trajectory marked with
\textcolor{red}{red-dashed line} concerns the results achieved without
early-switch, and the \textcolor{blue}{solid-blue line} is related to ones with
early-switch. The shadow regions concern the constraints of each mode. Also,
Fig.~\ref{fig:states} shows the borders of the estimates of the regions of
attraction, RoA, for each mode, with dot-dashed lines.

It is clear that the path taken under each algorithm's variation (with and
without early controller switch) are almost the same. The respective control
signals are given in Fig.~\ref{fig:control-signals}, pointing to no relevant
difference between the methods.

\begin{figure}[ht!]
  \centering
  \captionsetup{justification=centering}
  \includegraphics[width=0.8\linewidth]{imgs/tanks-control-signal}
  \caption{Control signals for Example 1.}%
  \label{fig:control-signals}
\end{figure}

\begin{figure}[ht!]
  \centering
  \captionsetup{justification=centering}
  \includegraphics[width=0.9\linewidth]{imgs/tanks-states}
  \caption{States trajectory for Example 1.}%
  \label{fig:states}
\end{figure}

\FloatBarrier

\subsection{Unstable System}%
\label{subsec:unstable-system}

Consider the unstable switching system with matrices
%
\begin{align*}
  A_1 & =
  \begin{bmatrix}
    1 & 0.003 \\
    0 & 1
  \end{bmatrix},
  A_2 = \begin{bmatrix}
    1 & 0.0074 \\
    0 & 1.1
  \end{bmatrix}, \\
  B_1 & =
  \begin{bmatrix}
    0.0005 & 1.2\times{}10^{-6} \\
    0      & 0.0008
  \end{bmatrix},
  B_2 = \begin{bmatrix}
    0.0019 & 3.6\times{}10^{-5} \\
    0      & 0.011
  \end{bmatrix}, \\
  %
  C_1 & = C_2 =
  \begin{bmatrix}
    1 & 0 \\
    0 & 1
  \end{bmatrix},       \\
\end{align*}
%
and operational points
%
\[
  x_{eq}^1 = \begin{bmatrix}
    1 \\ 1
  \end{bmatrix},
  u_{eq}^1 = \begin{bmatrix}
    -2 \\ \frac{-5}{4}
  \end{bmatrix},
  x_{eq}^2 = \begin{bmatrix}
    -1 \\ 1
  \end{bmatrix},
  u_{eq}^2 = \begin{bmatrix}
    \frac{-30}{19} \\ -10
  \end{bmatrix}.
\]
%
The sample-period was \SI{0.1}{\second}, and controller and observer gains are
given by:
%
\scriptsize
\begin{align*}
  K_1 & = \begin{bmatrix}
    -2.669\times{}10^3   & -1.993            & -6.741\times{}10^2    & 1.010             \\
    3.582\times{}10^{-4} & -1.669\times{}10^3 & -3.103\times{}10^{-4} & -4.210\times{}10^2
  \end{bmatrix}, \\
  %
  K_2 & = \begin{bmatrix}
    -7.034\times{}10^2    & -1.268            & -1.769\times{}10^2    & 6.097\times{}10^{-1} \\
    -3.903\times{}10^{-6} & -1.370\times{}10^2 & -1.292\times{}10^{-5} & -3.202\times{}10^1
  \end{bmatrix}, \\
  L_1 & = \begin{bmatrix}
    0.89374311 & 0.01193574 \\
    0.00922275 & 0.91488323
  \end{bmatrix},
  L_2 = \begin{bmatrix}
    0.64130679 & 0.0107965 \\
    0.00533561 & 0.8518838
  \end{bmatrix}.
\end{align*}
\normalsize

With the same procedures and considerations of the previous example, including
the used color codes, we simulated the unstable system.
Figure~\ref{fig:unstable-states} shows the same system trajectory for both
methods.

\begin{figure}[ht!]
  \centering
  \captionsetup{justification=centering}
  \includegraphics[width=\linewidth]{imgs/unstable_states}
  \caption{States trajectory for Example 2}%
  \label{fig:unstable-states}
\end{figure}

The second method shows better performance under control signal restrictions.
Figure~\ref{fig:unstable-control-signals} reveals a difference in the control
signals, where the first method, displayed in \textcolor{red}{red dashed-line}
and \textcolor{green}{green dashed-line}, have higher control signal outputs
than the second method, shown by \textcolor{blue}{blue solid-line} and
\textcolor{orange}{orange solid-line}.

\begin{figure}[ht!]
  \centering
  \captionsetup{justification=centering}
  \includegraphics[width=\linewidth]{imgs/unstable_control_signal}
  \caption{Control signals for Example 2}%
  \label{fig:unstable-control-signals}
\end{figure}

Both methods results in the same settling time, but with much lower control
effort with the strategy of early switching.

\FloatBarrier

\subsection{Cessna 182}%
\label{subsec:cessna}

\textcite{franzè.lucia.ea:command} present a Cessna 182 aircraft model, which
they used to simulate the dwell-time technique they proposed. They describe the
model, the operation points they chose, and the constraints applied to it. As
with the tank, we used the same system changing only the switch technique to
compare the performance. The proposed technique took \SI{6}{\second} to
converge, whereas the dwell-time technique took over \SI{20}{\second}.

Figure~\ref{fig:cessna-u} shows both actuators' control signals, which are kept
mostly invariant, except at the mode transitions, where the integrator reset
created a discontinuity. Figure~\ref{fig:cessna-x} shows each state. In both
figures, the solid black lines at the plot's top and bottom are the signal's
constraints. The last state (\(x_{4}\)) is the output, and its plot also shows
the reference and command governor's output.

\begin{figure}[ht!]
  \centering \captionsetup{justification=centering}
  \includegraphics[height=0.3\textheight]{imgs/cessna-u}
  \caption{Cessna simulation control signal}%
  \label{fig:cessna-u}
\end{figure}

\begin{figure}[ht!]
  \centering \captionsetup{justification=centering}
  \includegraphics[height=0.6\textheight]{imgs/cessna-x}
  \caption{Cessna simulation states}%
  \label{fig:cessna-x}
\end{figure}

\section{Experimental Results}%
\label{sec:experimental-results}

Consider an interactive tank system as indicated in Figure~\ref{fig:tanks}. It
describes the physical system present at the System Analysis Laboratory of
CEFET-MG campus V, shown in Figure~\ref{fig:tanks-real}. It consists of two
coupled tanks, \(T1\) and \(T2\), that are fed by a pump with controlled flow
\(u(t)\), measured in \si{\cubic\centi\metre\per\second}. The levels of each
tank, \(h_1\) and \(h_2\) (\si{\centi\metre}), are the control objective
variables, and are measured directly~\parencite{franco.oliveira.ea:síntese}.

\begin{figure}[ht!]
  \centering \captionsetup{justification=centering}
  \includegraphics[height=.5\textheight]{imgs/tanks-real}
  \caption{Tank system}%
  \label{fig:tanks-real}
\end{figure}

\begin{figure}[ht!]
  \centering \captionsetup{justification=centering}
  \includegraphics[width=0.9\linewidth]{imgs/tanks}
  \caption{Tank system diagram}%
  \label{fig:tanks}
\end{figure}

Both tanks have the same cross-section area, denoted as \(A\)
(\si{\square\centi\metre}), however, there is a solid inside \(T1\) that makes its
area non-linear and adds uncertainties to its model. The cross-section area of
\(T1\) becomes
%
\begin{equation}
  \label{eq:t1-area}
  A_{1}(h_{1}(t)) = \frac{3r}{5} \left(
  2.7r - \frac{\cos(2.5\pi{}((h_{1}(t)-8)\times{}10^{-2}-\mu))}{\sigma{}\sqrt{2\pi}}
  e^{-\frac{((h_{1}(t)-8)\times{}10^{-2}-\mu^{2})^{2}}{2\sigma^{2}}}
  \right),
\end{equation}
%
where \(\mu=0.4\), \(\sigma=0.55\) and \(r=0.31\). The cross-section area of \(T2\) is
\(\SI{0.31}{\square\metre}\).

By using Bernoulli's equations, the system's dynamics can be described by:
%
\begin{equation}
  \label{eq:formula-height-variation}
  \begin{aligned}
    \dot{h}_1(t) & = \frac{R_{12}(h_{1}(t),h_{2}(t))\times{}K_{b}\times{}u(t)-h_{1}(t)+h_{2}(t)}
    {A_{1}(h_{1}(t))\times{}R_{12}(h_{1}(t),h_{2}(t))}                                                                 \\
    \dot{h}_2(t) & = \frac{h_{1}(t)-h_{2}(t)}{R_{12}(h_{1}(t),h_{2}(t))\times{}A_{2}} - \frac{q_{o}(h_{2}(t))}{A_{2}},
  \end{aligned}
\end{equation}
%
where \(R_{12}(h_{1}(t),h_{2}(t))=(0.412(h_{1}(t-h_{2}(t))+11.488)\times{}10^{-3}\)
and \(q_{o}(h_{2}(t))=11.941h_{2}(t)+787.586\). A frequency inverter controls
the pump through the percentage of maximum flow, and this value can be converted
to flow by \(q_{i}=13.201u_{t}+220.085\). Applying this before inputing in
Equation~\eqref{eq:formula-height-variation} results in \(u(t)\) becoming this
percentual value instead of the flow directly.

We chose four operation points to cover a significant portion of the available
\(h_{1}\) range (0 to \SI{70}{\centi\metre}), which is the output of the system.
The points and their respective linearized and discretized (with a sampling time
of \SI{5}{\second}) systems are

\begin{align}
  \label{eq:op-points}
  \left[\begin{array}{c|c}
      x_{eq}^{\top} & u_{eq} \\
      \hline
      A             & B
    \end{array}\right]_{1} & = \left[\begin{array}{cc|c}
      19.5  & 5    & 15     \\
      \hline
      0.91  & 0.14 & 0.028  \\
      0.085 & 0.9  & 0.0013
    \end{array}\right], \\
  \left[\begin{array}{c|c}
      x_{eq}^{\top} & u_{eq} \\
      \hline
      A             & B
    \end{array}\right]_{2} & = \left[\begin{array}{cc|c}
      27.5  & 11.6 & 20     \\
      \hline
      0.94  & 0.19 & 0.04   \\
      0.084 & 0.9  & 0.0018
    \end{array}\right], \\
  \left[\begin{array}{c|c}
      x_{eq}^{\top} & u_{eq} \\
      \hline
      A             & B
    \end{array}\right]_{3} & = \left[\begin{array}{cc|c}
      37.3  & 17.8 & 25     \\
      \hline
      1.1   & 0.47 & 0.1    \\
      0.086 & 0.92 & 0.0042
    \end{array}\right], \\
  \left[\begin{array}{c|c}
      x_{eq}^{\top} & u_{eq} \\
      \hline
      A             & B
    \end{array}\right]_{4} & = \left[\begin{array}{cc|c}
      47.4  & 24.9 & 30     \\
      \hline
      0.49  & 0.96 & 0.22   \\
      0.053 & 0.95 & 0.0096
    \end{array}\right].
\end{align}

To develop the controllers we applied the LMI described in the
Section~\ref{sec:region-of-attraction}~-~\nameref{sec:region-of-attraction} to
the integral-augmented system. The augmented state is the integral of the
output, \(h_{1}\). The points forced to be inside the region of attraction are
the previous' and next's operation point. It creates an intersection that allows
the early switching of controllers, which speeds up convergence. The following
controllers were obtained:

\begin{align}
  K_{1} & = \begin{bmatrix} -12.884 & -97.540 & -13.975 \end{bmatrix}, \\
  K_{2} & = \begin{bmatrix} -10.054 & -73.777 & -10.523 \end{bmatrix}, \\
  K_{3} & = \begin{bmatrix} -5.840  & -31.622 & -4.148 \end{bmatrix}, \\
  K_{4} & = \begin{bmatrix} -1.832  & -21.527 & -4.177 \end{bmatrix}.
\end{align}

To calculate each of the described modes' dwell-time, we applied the technique
described by~\textcite{franzè.lucia.ea:command}. We then ran both approaches ---
the dwell-time and the proposed one --- on the physical system.
Figures~\ref{fig:tanks-dwell} and~\ref{fig:tanks-roa} show the result of the
experiments, where the colored backgrounds represent the different modes, M1 to
M4, the two states (\(x_{1}\) and \(x_{2}\)) are plotted alongside their
respective real (\(r_{1,2}\)) and virtual (\(g_{1,2}\)) references.

The proposed technique has a faster overall convergence, entering the final mode
after 215 seconds while the dwell-time approach takes 340 seconds. This can
easily be seen by comparing the width of the regions with different background
colors, as the red, orange and yellow regions are much shorter on the proposed
technique. The proposed approach also shows no overshoot, since it does not give
the system enough time to converge to the waypoints and keeps it always moving
towards the reference, as shown by the states' trajectories, which are always
moving torwards the reference in the proposed technique, but starts to converge
to the mode's operation point on the dwell-time technique. Because of that, it
also has a smaller control effort, producing smaller control signal variation,
seen in the last plot, where the control signal constantly saturates in the
dwell-time approach, but does not saturate and has a smaller variation in the
proposed technique.

The last mode should also have an overshoot on the proposed technique. However,
the command governor's optimization problem generated virtual references that
slowly guided the system to the real reference, making it take longer than
necessary to converge. The behaviour was present on every test repetition, just
like the dwell-time approach's overshoots.

\begin{figure}[ht!]
  \centering \captionsetup{justification=centering}
  \includegraphics[width=\linewidth]{imgs/tanks-dwell}
  \caption{Dwell-time approach}%
  \label{fig:tanks-dwell}
\end{figure}

\begin{figure}[ht!]
  \centering \captionsetup{justification=centering}
  \includegraphics[width=\linewidth]{imgs/tanks-roa}
  \caption{Proposed RoA-based approach}%
  \label{fig:tanks-roa}
\end{figure}
