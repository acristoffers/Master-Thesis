% !TeX root = document.tex
% !TeX encoding = UTF-8 Unicode

\chapter{Final Considerations}%
\label{chp:final-considerations}

Switched systems usually lose stability under arbitrary switching, requiring the
development of switching rules that keep the system stable after the switch. A
class of switching rules denominated slow-switching guarantees stability by not
switching between modes too fast. Most of such techniques try to determine the
minimum amount of time the system needs to remain on a mode before switching to
remain stable.

Command Governors enforces constraints on dynamic systems. When used with
switched systems, it enforces constraints on each mode, aggravating the
stability problem. Now the system is not allowed to switch from any state, only
from a subset inside the constraints' intersection. The design of slow-switching
rules must take this into account to avoid violating constraints.

In this work, we studied switched systems' constraints enforcement through
Command Governors. The switched systems' stability problem, usually solved using
dwell time, was reviewed under a new light, resulting in the proposed Region of
Attraction-based switching rule, something not found in literature. The rule,
shown in Algorithm~\ref{alg:roa-rule}, guarantees the system stability after
mode switches by only allowing the switch if the system's states remain inside
the controller's Region of Attraction and are inside constraints' intersection.

On the same algorithm, we can switch the controller earlier while keeping the
constraints, which we call a hybrid change. It allows for faster convergence
when the next controller performs better than the current one. However, if that
is undesirable, the switch can happen at once on the constraints' intersection,
still having all stability guarantees from the Region of Attraction.

The presented technique allows faster overall convergence of constrained
switched systems while ensuring stability after switches. It is based on
well-consolidated concepts and builds on them to provide stability to
constrained switched systems. Although it was not designed to be used without a
Command Governor, the rule can be further extended to guarantee stability in
that case as well.

The major problem with the technique is that it is hard to apply it to existing
controllers since their region of attraction might not satisfy the necessary
criteria of size and intersections. Because of this, the main advantage of the
Command Governor is lost and new controllers will most probably need to be
synthetized.

On Chapter~\ref{chp:results} we apply the technique on different simulated
systems as well as a real one. The simulations illustrate how the proposed rule
works and compares it with results from an article. The experimental result
shows the convergence time gain when compared to the dwell-time technique as
well as the better control signal trajectory. The simplicity of the algorithm
does not put a burden in the hardware implementation and the design requires the
estimation of the Region of Attraction where other slow-switch techniques
require the estimation of the dwell-time.

This work resulted in the publication of an article at Congresso Brasileiro de
Automática --- CBA in November 2020 \parencite{sousa.silva.ea:command}. An
extended version, including the experimental results obtained, is being prepared
to be submitted to a journal.

\section{Perspectives}%
\label{sec:perspectives}

This work only deals with LTI systems. Future research can extend the technique
to models with uncertainties, varying the region of attraction or modifying the
switching rule to accommodate them. In this case, the designer must pay
attention to the possibility of the system becoming unstable if the switch
happens when the system is not inside the region of attraction due to model
mismatch, possibly leading to instability.

On those lines, switched systems composed of LPV subsystems also need special
attention. In this case, the Region of Attraction can also become a function of
the states, and special attention might be necessary to ensure there are
intersections.

Last but not least, the extension of the proposed rule to non-command governor
schemes also needs consideration. The Command Governor plays a role in not
letting the system destabilize, so it becomes necessary in the proposed schema.
