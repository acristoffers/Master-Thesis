% !TeX root = document.tex
% !TeX encoding = UTF-8 Unicode

\chapter{Final Considerations}%
\label{chp:final-considerations}

We presented two new schemes for Control Governor switching that do not use
dwell-time, leading to faster closed-loop responses. When the control's
objective is to take the system from one operational condition to another, it
might not be necessary to calculate a dwell-time. A dwell-time is only necessary
to prevent the system from switching back and forth and becoming unstable, which
does not happen when the system switches and moves away from the switching
condition.

In the presented cases, the switch can only occur at the intersection of the
constraints. The supervisor will move the reference to the next way-point as
soon as the CGs are swapped, disqualifying any further CG change. It is then
only necessary to show the stability of the switch.

It is possible to calculate the region of attraction of the controller designed
using the Lyapunov theorem. It is then possible to guarantee that, once
switched, the system will converge to the new reference. Because of this, both
methods avoid waiting for the system to converge to a way-point. The advantage
is a shorter time to reach the real reference.

The proposed method allow for faster mode switching when the system can not, by
design, go back to the previous mode. The approach allow us to bypass the
dwell-time and achieve better closed-loop performance indexes, such as the
settling time. The use of the region of attraction can be further investigated
to easy the work done by the command governor, by exploring the contractivity
properties of the estimate region of attraction.
