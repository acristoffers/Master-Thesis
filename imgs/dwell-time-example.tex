\begin{figure}[ht!]
  \centering
  \begin{tikzpicture}[auto,node distance=3cm,>={Stealth},waypoint/.style={draw,circle,minimum size=1em,inner sep=0pt,outer sep=0pt,thick},state/.style={draw,thick,circle,minimum size=1em,inner sep=0pt,outer sep=0pt},constraint/.style={ellipse,fill opacity=0.7,text opacity=1}]
    \node (x)  [state]                {\(\bullet\)};
    \node (w1) [waypoint,below=of x]  {\(\diamond\)};
    \node (w2) [waypoint,right=of w1] {\(\diamond\)};
    \node (w3) [waypoint,above=of w2] {\(\star\)};

    \begin{scope}[on background layer]
      \node (c1) [constraint,fill=cyan!80,fit=(x) (w1)]    {};
      \node (c2) [constraint,fill=green!80,fit=(w1) (w2)]  {};
      \node (c3) [constraint,fill=orange!80,fit=(w2) (w3)] {};
    \end{scope}

    \draw [->,thick] (x) -- (w1);
    \draw [->,thick] (w1) -- (w2);
    \draw [->,thick] (w2) -- (w3);
  \end{tikzpicture}%
  \caption[Dwell-time illustrative example.]{Dwell-time illustrative example.
    The plane is the phase-plane of a second order system. Each colored region
    represents one mode's contraints, each symbol inside a circle represents a
    point of interest: \(\bullet\) is the initial state, \(\diamond\) are the waypoints and
    \(\star\) the final reference, as set by the operator. The arrows show the path
    the system will take as the supervisor changes the references to the
    waypoints and then to \(\star\).}%
  \label{fig:dt-example}
\end{figure}
