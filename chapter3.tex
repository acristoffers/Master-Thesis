% !TeX root = document.tex
% !TeX encoding = UTF-8 Unicode

\chapter{Switching Rules}%
\label{chp:switching-rules}

This chapter presents two switching rules: the dwell-time, the most employed and
studied of such techniques in literature, and the proposed rule based on the
region of attraction. Other techniques exist, like using robust control to find
a controller that never makes the system unstable; however, they are more
conservative and not viable in many cases.

\section{Dwell-time}%
\label{sec:dwell-time}

As discussed in Section~\ref{sec:switched-systems}, the act of switching can
cause instability, even if all modes are stable. Although it is possible to
verify if the system is stable under arbitrary switching (and therefore to
design controllers to do so), the procedure is not straight-forward and
challenging to apply to most real-world situations.

There are, however, other ways of verifying and guaranteeing the stability of
systems under switches governed by some rules. The dwell-time is one of such
techniques that restricts the switching signal \(\sigma{}(t)\) to the set
%
\begin{equation}
  \mathcal{D}_{T} := \{\sigma(\cdot):t_{k+1}-t_{k}\ge{}T\},
\end{equation}
%
where \(t_{k}\) and \(t_{k+1}\) are the switching instants, for all
\(k\in{}\mathbb{R}\), which forces the system to remain for \(T\) seconds on a
mode before switching to next one~\parencite{colaneri:dwell}. This is called a
slow switch. The timer is restarted every time the reference changes and
switching is only allowed after \(T\) seconds has passed. For large enough
values of \(T\), this rule guarantees the system's stability.

As the dwell-time certifies stability of the switch, it decouples the switching
logic and the system stability, making it possible to analyse the system
stability for each mode independently. One problem, however, is that computing
the minimum dwell-time is not easy and is the focus of current research. An
easier problem is to find an upper bound for it, which can be done efficiently
using numerical algorithms~\parencite{colaneri:dwell}.

Another technique that uses the concept of dwell-time is the average dwell-time,
where the switching rule \(\sigma\) allows for a fixed number of discontinuities
\(N_{\sigma}(t,\tau)\) for \(t\ge{}\tau{}\ge{}0\) such that the set
\(\mathcal{D}_{T_{D},N_{0}}\) satisfies
%
\begin{equation}
  N_{\sigma}(t,\tau) \le{} N_{0} + \frac{t-\tau}{\tau_{D}},
\end{equation}
%
where \(\tau_{D}\) is the average dwell-time and \(N_{0}\) is the chatter
bound~\parencite{hespanha.morse:stability-of-switched-systems-with-average-dwell-time}.
This set is larger than \(\mathcal{D}_{T}\) and allows for signals with
discontinuities separated by at most \(\tau_{D}\).

To illustrate how the dwell-time works, consider the fictional system if
Figure~\ref{fig:dt-ex1}. The ellipses represent each mode's constraint region,
the black star is the system state, the orange star (in the middle) is the
waypoint and the yellow star the reference. The system will use the dwell-time
rule to switch.

\begin{figure}[!htb]
  \centering
  \includesvg[width=0.5\linewidth]{imgs/dt-ex1}
  \caption{System to be considered in the example}%
  \label{fig:dt-ex1}
\end{figure}

To avoid constraint violation, the reference is first set to the waypoint by the
supervisor. As the reference changed, the timer will be restarted and the system
will only be allowed to change mode after the dwell-time ellapses. However, the
system may fully converge to the waypoint before that, making it just wait,
idle, for the timer to ellapse, as depicted in Figure~\ref{fig:dt-ex2}.

\begin{figure}[!htb]
  \centering
  \includesvg[width=0.5\linewidth]{imgs/dt-ex2}
  \caption{First state's path: convergence to the waypoint}%
  \label{fig:dt-ex2}
\end{figure}

The system will change modes after the timer ellapses and the reference will be
set to the desired reference, in yellow. Because the reference was changed
again, the timer will restart again, with the dwell-time of the second mode.
Figure~\ref{fig:dt-ex3} shows this movement.

\begin{figure}[!htb]
  \centering
  \includesvg[width=0.5\linewidth]{imgs/dt-ex3}
  \caption{Second state's path: convergence to the final reference}%
  \label{fig:dt-ex3}
\end{figure}

As it is the final reference, this timer will not affect the overall
performance, however, if there were more modes to go through, their dwell time
would be summed, and the total convergence time would be at least the sum of all
dwell-times that need to be executed on the path to the final reference.

Algorithm~\ref{alg:dwell-time} formalizes this procedure.

\begin{algorithm}[H]
  \begin{algorithmic}[1]
	\State{}\textbf{Input}: \(\CG{}_i\) \(\leftarrow{}\) current \CG{},~\(\CG{}_j\) \(\leftarrow{}\) target GG.\@
	\State{}change \(r(k)\) to next way-point
	\While{dwell-time of \(\CG{}_{i}\) not ellapsed}
		\State{}calculate \(g(k)\)
		\State{}execute controller
	\EndWhile{}
	\State{}change to \(\CG{}_j\)
	\State{}restart algorithm
  \end{algorithmic}
  \caption{dwell-time implementation}%
  \label{alg:dwell-time}
\end{algorithm}

\section{Region of Attraction}%
\label{sec:roa-switching-rule}

The goal of this switching rule is to allow the system to converge faster to the
final reference when going through a path of restricted mode switches. To do so,
it is necessary to:

\begin{enumerate}
  \item guarantee stability after switching modes,
  \item switch modes as soon as possible.
\end{enumerate}

To guarantee stability, the concept of a controller's region of attraction is
used. As described in Section~\ref{sec:region-of-attraction}, the controller's
region of attraction is a closed region in the state-space such that every
initial state inside it will remain inside it as well as asymptotically converge
to a point inside it (usually the origin of the linearized system).

The region of attraction is, therefore, a certificate of stability for a system.
However, as stated in Section~\ref{sec:switched-systems}, having stable modes is
not enough to guarantee the stability of the system after switching.
Nonetheless, the following region of attraction based rule guarantees that the
system will remain stable after switching:

\begin{align}
  \sigma_{i} = \begin{cases}
    1 & \textrm{if}~f\hat{\xi}_{i}(k)\in\mathcal{L}_V(P_i) \\
    0 & \textrm{otherwise}
  \end{cases},
\end{align}
%
where \(\hat{\xi}_{i}(t)\) is the mode's state, \(\mathcal{L}_V(P_i)\) is the
region of attraction (level set of the \(P_{i}\) matrix from the Lyapunov
function), and \(\sigma_{i}\) is the switching rule, meaning that the system can
switch to the \(i\)th mode if \(\sigma_{i}=1\).

Let us discuss the stability of the system when this rule is used. Suppose a
system is currently in mode 1, and the supervisor has deemed that it needs to go
mode 2. The switch will only occur when the system's state is inside the second
mode's controller's region of attraction. Because it is inside the region of
attraction, it guarantees that it will converge, and therefore the switch is
stable.

Now consider the same scenario, but the mode change follows the path
\(1 \rightarrow 3 \rightarrow 8 \rightarrow 2\), meaning that, to go from \(1\) to \(2\), it first needs to
switch to \(3\), then to \(8\) and only then to \(2\). The worst-case would be
if the system is in the middle of the intersection of all regions of attraction.
In this case, the system will instantly switch modes, going from \(1\) to \(2\)
in \(3\) sample times, in the case of a discrete system. However, since the
switch can only happen when the state is inside the next mode's controller's
region of attraction, it will simply cause the previous scenario to be applied
recursivelly, and the system will remain stable after it reaches the last mode.

In another scenario, the supervisor might malfunction or have a badly defined
rule that will make the reference jump between many values, making the system
try to switch modes seemly randomly. In this case the system will do the
switches as long as it is allowed, but the state will always be inside some
controller's region of attraction. It will then either never converge nor
diverge (because of the ever-changing reference) or enter a state that is only
inside one mode's controller's region of attraction, at which point it will not
switch modes anymore.

The illustrative scenarios show that the state will always be inside some
controller's region of attraction, and therefore it is not possible for the
system to diverge. It can oscilate due to varying references, but that is a case
of malfunction or badly desined referencing system, not normal operation.
Furthermore, when used in conjunction with Command Governors, the reference is
always guaranteed to be inside the region of attraction (since the constraint
region is completely contained inside it), even if the supervisor's reference is
not, eliminating the possible scenario where the system diverges because the
reference is outside the region of attraction, dragging the system out of it.

To visually illustrate the technique, consider the fictional system depicted in
Figure~\ref{fig:roa-ex1}, where the black star is the system's state, the yellow
star is the supervisor's reference, the orange star is the waypoint, the filled
ellipses are the mode's constraints and the circles are the regions of
attraction.

\begin{figure}[!htb]
  \centering
  \includesvg[width=0.45\linewidth]{imgs/roa-ex1}
  \caption{System to be considered in the example}%
  \label{fig:roa-ex1}
\end{figure}

As the system cannot violate constraints, the supervisor will set the reference
to the waypoint, and the system will converge to it. However, as soon as it
crosses the next mode's controller's region of attraction (or after some
distance is travelled inside it, just to be safe), as shown if
Figure~\ref{fig:roa-ex2}, it can switch modes without switching the constraints,
what we call a hybrid switch. This is opitional, but may yield better overall
performance.

\begin{figure}[!htb]
  \centering
  \includesvg[width=0.45\linewidth]{imgs/roa-ex2}
  \caption{System's state crossing the region of attraction's border}%
  \label{fig:roa-ex2}
\end{figure}

The system will then continue to converge to the waypoint, since it can only
change to the next reference once it is inside the constraint's intersection. As
soon as it enters the intersection the full mode change will be performed,
changing the current constraint to that of the next mode and setting the
reference to the global one, represented by the yellow star. Note that the
system does not fully converge to the waypoint. This step is shown in
Figure~\ref{fig:roa-ex3}.

\begin{figure}[!htb]
  \centering
  \includesvg[width=0.45\linewidth]{imgs/roa-ex3}
  \caption{System's state crossing the constraint's intersection's border}%
  \label{fig:roa-ex3}
\end{figure}

With this switching rule, there is no wait: the system will change mode as soon
as possible, yielding faster overall convergence. The hybrid switching is an
addition that may increase the performance of the system at the cost of
computation resources, but is not necessary and can be safelly ignored.
Algorithm~\ref{alg:roa-rule} shows the proposed rule in both its forms, assuming
the use of a Command Governor, which is always recommended for this rule as it
helps to guarantee stability.

\begin{algorithm}[H]
  \begin{algorithmic}[1]
    \State{}\textbf{Input}: \(\CG{}_i\) \(\leftarrow{}\) current \CG{},~\(\CG{}_j\) \(\leftarrow{}\)
        target GG.\@
    \State{}let \(P_1\) and \(P_2\) be regions of attraction for \(\CG{}_i\)
        and \(\CG{}_j\), respectively.
    \If{should switch controllers earlier}
      \While{\(\hat{\xi}(k)\not\in\mathcal{L}_V(P_2)\)}
        \State{}calculate \(g(k)\)
        \State{}execute controller
      \EndWhile{}
      \State{}change current controller and model to \(\CG{}_j\)'s ones
      \State{}reset integrators
    \EndIf{}
    \While{\(x(k)\not\in\mathcal{X}_i\cap\mathcal{X}_j\) or \(\hat{\xi}(k)\not\in\mathcal{L}_V(P_2)\)}
      \State{}calculate \(g(k)\)
      \State{}execute controller
    \EndWhile{}
    \State{}change to \(\CG{}_j\)
    \State{}reset integrators if not already done
    \State{}change \(r(k)\) to next way-point
    \State{}restart algorithm
  \end{algorithmic}
  \caption{Switching rule based on region of attraction}%
  \label{alg:roa-rule}
\end{algorithm}
